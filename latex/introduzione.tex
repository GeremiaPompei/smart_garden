\chapter{Introduzione}

\textbf{OpenHAB} sta per {\em Open Automation Bus} ed è una piattaforma di automazione domestica. Come dice il nome è {\em open source} ed il suo punto di forza è l'indipendenza dalla tecnologia; infatti i dispositivi intelligenti possono connettersi a prescindere dalla marca o dal tipo di device. Tale software installato in un dispositivo che funge da server permette di controllare l'intera casa intelligente. Inoltre {\em openHAB} fornisce un approccio comune alle regole di automazione dell'intero sistema così da rendere più immediato e semplice l'utilizzo. Punti di forza sono anche la flessibilità nell'integrazione di molti prodotti smart e l'alto grado personalizzazione.

\section{Motivazione}
Il motivo che porta allo studio e approfondimento di {\em openHAB} \'e soprattutto la sua caratteristica ad essere molto versatile alla configurazione con diversi dispositivi. Ci\'o permette quindi di abbinarlo in diversi contesti con elementi totalmente differenti tra loro. In questo caso verr\'a configurato per dialogare con un sistema IOT totalmente sviluppato da zero. Avendo come scopo principale quello di essere dinamico sar\'a un gioco da ragazzi controllare dispositivi sviluppati autonomamente in maniera semplice ed efficace tramite {\em openHAB}.

\section{Obiettivi}
L'obiettivo di questa tesi \'e studiare e approfondire openHAB per poi configurarlo e andarlo ad usare con un piccolo sistema IOT chiamato {\em Smart Garden} per il controllo del proprio giardino in maniera centralizzate insieme agli altri sistemi High Tech disponibili nella propria abitazione. Ci\'o permetter\'a in un futuro momento anche l'interazione di tale sistema IOT con altri configurando regole cos\'i da rendere la propria casa il pi\'u smart e comoda possibile.

\section{Struttura della Tesi}
La Tesi inizier\'a con l'analisi del framework entrando nei dettagli della propria logica interna definendo i vari elementi e l'interazione con essi. Si approfondiranno quindi aspetti legati alla modellazione base di {\em openHAB} e come questa pu\'o essere applicata alla domotica casalinga oltre alla comunicazione con il framework dall'esterno tramite un protocollo di rete.

Successivamente si approfondir\'a lo studio sul {\em protocollo MQTT} riguardo a come funziona e quali sono gli elementi al suo interno. Ci\'o sar\'a utile per la sezione successiva relativa all'associazione di {\em openHAB} tramite tale protocollo. Qu\'i verr\'a spiegato come viene vista l'associazione nel framework e come pu\'o essere configurata in base alle tipologie di elementi messi gi\'a a disposizione da esso.

Infine si parler\'a del sistema IOT sviluppato e di come questo interagisce con openHAB. Si approfondir\'a l'idea di base e lo sviluppo dietro lo {\em Smart Garden} e la sua associazione con il protocollo MQTT. In seguito verr\'a approfondita la configurazione di {\em openHAB} per dialogare con il sistema IOT.